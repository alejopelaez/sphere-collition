\documentclass[10pt,letterpaper]{article}
%\usepackage[latin1]{inputenc}
%\usepackage{amsmath}
\usepackage{amsfonts}
\usepackage{amssymb}
%Español y tildes!
\usepackage[utf8]{inputenc}
\usepackage[spanish]{babel}
\usepackage{listings}
\usepackage[usenames,dvipsnames]{color}
\usepackage{amsmath}
\usepackage{verbatim}
\usepackage{hyperref}
%%%%%%%%%%%%%%%%%%%%%%%%%%%%%%
\begin{document}
\title{Manual De Usuario}
\author{Nicolás Hock - Cristian Isaza - Alejandro Peláez}
\date{\today}
\maketitle
%%%%%%%%%%%%%%%%%%%%%%%%%%%%%%
\tableofcontents
%\lstlistoflistings
\lstloadlanguages{C++}
%%%%%%%%%%%%%%%%%%%%%%%%%%%%%%
\section{Descripción de la práctica}
\subsection{Información General}
\mbox{} \\
\mbox{La práctica pretende simular una colisión de esferas al interior de una caja. Las esferas} \\
\subsection{Criba de Eratóstenes}
\mbox{} \\
\input{./src/teoria_numeros/criba}%.tex
\subsection{Divisores de un número}
Saca los divisores de un número en desorden. Complejidad O($\sqrt{n}$).
Hasta 4294967295 (máximo \textit{unsigned long}) responde instantaneamente. Se puede
forzar un poco más usando \textit{unsigned long long} pero más allá de $10^{12}$ empieza a
responder muy lento.
\mbox{} \\
\mbox{} \\
\input{./src/teoria_numeros/divisores}%.tex
\subsection{Función $ \varphi $ de Euler}
Si n es un número entero positivo, entonces $ \varphi $(n) se define como el número de enteros positivos menores o iguales a n y coprimos con n.
\mbox{} \\
\mbox{} \\
$$ \varphi (n) = \left\{
\begin{array}{c l}
 n-1 & \mbox{Si n es primo}\\
 \displaystyle n\prod_{p|n}\left(1-\frac{1}{p}\right) & \mbox{donde los p son los distintos primos que dividen a n.}\\
\end{array}
\right.
$$
\mbox{} \\
\mbox{} \\
\input{./src/teoria_numeros/phiEu}%.tex
\subsection{Fibonacci}
Calcula el fibonacci N en O(log n).
\mbox{} \\
\mbox{} \\
\input{./src/teoria_numeros/fib}%.tex
\subsection{GCD}
Algoritmo de Euclides
\mbox{} \\
\mbox{} \\
\input{./src/teoria_numeros/gcd}%.tex
%%%%%%%%%%%%%%%%%%%%%%%%%%%%%%
\section{Combinatoria}
\subsection{Cuadro resumen}
Fórmulas para combinaciones y permutaciones:
\begin{center}
\renewcommand{\arraystretch}{2} %Multiplica la altura de cada fila de la tabla por 2
%Si quiero aumentar el tamaño de una fila en particular insertar \rule{0cm}{1cm} en esa fila.
\begin{tabular}{| c | c | c |}
\hline
\textit{Tipo} & \textit{¿Se permite la repetición?} & \textit{Fórmula} \\ [1.5ex]
\hline\hline

$r$-permutaciones & No & $ \displaystyle\frac{n!}{(n-r)!} $ \\ [1.5ex]
\hline
$r$-combinaciones & No & $ \displaystyle\frac{n!}{r!(n-r)!} $ \\  [1.5ex]
\hline
$r$-permutaciones & Sí & $ \displaystyle n^{r} $ \\
\hline
$r$-combinaciones & Sí & $ \displaystyle\frac{(n+r-1)!}{r!(n-1)!} $ \\ [1.5ex]
\hline
\end{tabular}
\renewcommand{\arraystretch}{1}
\end{center}
Tomado de \textit{Matemática discreta y sus aplicaciones}, Kenneth Rosen, 5${}^{\hbox{ta}}$ edición, McGraw-Hill, página 315.
\subsection{Combinaciones, coeficientes binomiales, triángulo de Pascal}
\emph{Complejidad:} $ O(n^2) $ \\
$$ {n \choose k} = \left\{
\begin{array}{c l}
 1 & k = 0\\
 1 & n = k\\
 \displaystyle {n - 1 \choose k - 1} + {n - 1 \choose k} & \mbox{en otro caso}\\
\end{array}
\right.
$$

\input{./src/combinatoria/pascal_triangle}%.tex

\bigskip 
\textbf{Nota:} $ \displaystyle {n \choose k }  $ está indefinido en el código anterior si $ n > k$. ¡La tabla puede estar llena con cualquier basura del compilador!

%%%%%%%%%%%%%%%%%%%%%%%%%%%%%%
\section{Grafos}
%%%%%%%%%%%%%%%%%%%%%%%%%%%%%%
\section{Programación Dinámica}
%%%%%%%%%%%%%%%%%%%%%%%%%%%%%%
\section{Geometría y Trigonometría}
%%%%%%%%%%%%%%%%%%%%%%%%%%%%%%
\section{Estructuras de Datos}
%%%%%%%%%%%%%%%%%%%%%%%%%%%%%%
\section{Otros}
%%%%%%%%%%%%%%%%%%%%%%%%%%%%%%
\section{Java}
%%%%%%%%%%%%%%%%%%%%%%%%%%%%%%
\section{C++}
%%%%%%%%%%%%%%%%%%%%%%%%%%%%%%
\end{document}